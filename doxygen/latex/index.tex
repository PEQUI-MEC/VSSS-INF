

\section*{Pequi Very Small}

\href{https://travis-ci.org/PEQUI-MEC/VSSS-INF}{\tt } \href{https://www.codacy.com/app/rafaelfgjardim/VSSS-INF?utm_source=github.com&amp;utm_medium=referral&amp;utm_content=PEQUI-MEC/VSSS-INF&amp;utm_campaign=Badge_Grade}{\tt }   \href{https://github.com/PEQUI-MEC/VSSS-INF/blob/master/docs/LICENSE}{\tt }

View this page on another language\+: \href{https://github.com/PEQUI-MEC/VSSS-INF}{\tt Portuguese} https\+://github.com/\+P\+E\+Q\+U\+I-\/\+M\+E\+C/\+V\+S\+S\+S-\/\+I\+N\+F/blob/master/docs/\+R\+E\+A\+D\+M\+E.\+en.\+md \char`\"{}\+English\char`\"{}

Olá! Esse é o repositório de desenvolvimento da equipe de robótica {\bfseries Very Small Size Soccer} do \href{https://www.facebook.com/NucleoPMec/}{\tt Pequi Mecânico}. Nosso grupo é composto por integrantes de várias cursos (Engenharia Elétrica, Engenharia da Computação, Engenharia de Software, Engenharia Florestal e Ciências da Computação) todos da Universidade Federal de Goiás -\/ \href{https://www.ufg.br/}{\tt U\+FG} -\/ Goiânia.

Nosso repositório é aberto pois entendemos que nosso maior trabalho é agregar nossas pesquisas e conhecimentos ao mundo acadêmico e comercial.

Estamos abertos a responder qualquer dúvida e sugestão através do nosso email \href{mailto:contato@pequimecanico.com}{\tt contato@pequimecanico.\+com}. Mais informações em nosso \href{https://pequimecanico.com/}{\tt S\+I\+TE}.

\section*{Como a usar}

\subsection*{Dependências}

Para começar a utilizar nosso software em primeiro lugar é necessário que tenha instalado alguma distro Linux, de preferência uma Debian-\/like. Caso tenha, basta executar nosso \href{https://github.com/PEQUI-MEC/VSSS-INF/blob/master/run.sh}{\tt run.\+sh}. O mesmo executará uma verificação de dependências e o que estiver faltando será instalado.

Caso queira instalar manualmente as dependências, segue a lista para instalação\+:


\begin{DoxyItemize}
\item build-\/essential
\item cmake
\item git
\item libgtk2.\+0-\/dev
\item pkg-\/config
\item libavcodec-\/dev
\item libavformat-\/dev
\item libswscale-\/dev
\item Opencv v3.\+4.\+1
\item Opencv Contrib v3.\+4.\+1
\item Libxbee3 v3.\+0.\+11
\item libboot-\/all-\/dev
\item libv4l-\/dev
\item libv4lconvert0
\item libgtkmm-\/3.\+0-\/dev
\end{DoxyItemize}

\subsection*{Gerando o executável}

Nosso sistema possui um arquivo C\+Make\+Lists.\+txt para construir nosso executável. Nosso script \href{https://github.com/PEQUI-MEC/VSSS-INF/blob/master/build.sh}{\tt build.\+sh} resolve automaticamente a construção da aplicação.

Caso deseje fazê-\/lo manualmente, dentro da pasta do projeto crie uma pasta chamada {\bfseries build}, abra o terminal dentro da pasta e execute o comando

$>$cmake ..

Em seguida execute

$>$make

O projeto gerará um executável na raiz do projeto com o nome de P137. Basta executá-\/lo.

\section*{Wiki}

Nossa \href{https://github.com/PEQUI-MEC/VSSS-INF/wiki}{\tt Wiki} pode conter informações relevantes para seu melhor entendimento de como trabalhamos e como você pode contribuir com nosso trabalho.

\section*{Redes Sociais}

Nossas atividades e eventos estão sempre em atualização através de nossos canais sociais. Fique por dentro de nossas atividades e se encante com o mundo da robótica.


\begin{DoxyItemize}
\item \href{https://www.instagram.com/pequimecanico/}{\tt I\+N\+S\+T\+A\+G\+R\+AM}
\item \href{https://www.facebook.com/NucleoPMec}{\tt F\+A\+C\+E\+B\+O\+OK} 
\end{DoxyItemize}